\input ../common.tex

\graphicspath{{fig/}}

\institute[Nick Papior; DTU Compute]{\begin{tikzpicture}
      \node[shape=rectangle split,rectangle split parts=2,anchor=base] at (0,0)
      {DTU: sisl, TBtrans and TranSiesta workshop};
    \end{tikzpicture}}

\date{17 May 2021}
\title{Siesta and sisl analysis}
\author{Nick Papior}

\begin{document}

\begin{frame}
  \titlepage
\end{frame}

\begin{frame}
  \frametitle{Outline}
  \tableofcontents
\end{frame}

\input siesta.tex
\input sisl.tex

\section{Conclusion}

\begin{frame}
  \frametitle{Conclusion}

  \begin{block}{Siesta}
    \begin{description}
      \item[+] ``All'' necessary tools
      \item[-] Utilites uses a variety of executables
      \item[-] Utilites uses a variety of input methods
      \item[-] Utilites are limited to the current sources (\emph{very} rarely changed/updated)
    \end{description}
  \end{block}

  \begin{block}{sisl}
    \begin{description}
      \item[-] Not all utilities are present
      \item[+] A unified interface for post-processing
      \item[+] Easy to extend due to simpler programming language
      \item[+] Many functionalities present, and some utilities has more flexibility
    \end{description}
  \end{block}
  
\end{frame}

\end{document}
