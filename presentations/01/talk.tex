\input ../common.tex

\graphicspath{{fig/}{../fig/}}
\usepackage[export]{adjustbox}

\institute[Nick Papior; DTU Compute]{\begin{tikzpicture}
      \node[shape=rectangle split,rectangle split parts=2,anchor=base] at (0,0)
      {DTU: sisl, TBtrans and TranSiesta workshop};
    \end{tikzpicture}}

\date{13 November 2023}
\title{Non-equilibrium Green function theory: 1 \& 2}
\author{Nick Papior}


\begin{document}

\begin{frame}
  \frametitle{Practical information}
  
  \small
  \begin{itemize}

    \item Lectures \emph{will} be recorded (by joining any session you agree to this)

    \item Shotgun presentation Tuesday morning at 9 ($\sim1$ min for each!)

      Please send me your presentation (2 slides MAX) \url{mailto:nickpapior@gmail.com}

    \item Social dinner Tuesday at 20 at Madklubben, please repeat dietary needs on paper!

    \item Please ask questions during the workshop, you are here to learn!
    
    \item Help each other out, we are all peers.
    
    \item This workshop is a loose format, primarily hands-on, so time-schedules are
      approximate, and meant for you to understand things better; please do not rush through the exercises!

 \end{itemize}

 \vskip 3em
 \begin{center}
   \large
   Siesta, TranSiesta, TBtrans and sisl are \emph{not} static codes! They evolve due to
   comments and contributions from users, so please contribute!
 \end{center}
 
\end{frame}

\begin{frame}
  \titlepage
\end{frame}

\begin{frame}
  \frametitle{Outline}
  \tableofcontents
\end{frame}

% TODO commented out:
%\input introduction.tex
\input green.tex
\input se.tex
\input negf.tex
\input options.tex
\input electrodes.tex
\input device.tex

\end{document}
